
\documentclass[a4paper,12pt]{article}

%%% Работа с русским языком
\usepackage{cmap}					% поиск в PDF
\usepackage{mathtext} 				% русские буквы в формулах
\usepackage[T2A]{fontenc}			% кодировка
\usepackage[utf8]{inputenc}			% кодировка исходного текста
\usepackage[english,russian]{babel}	% локализация и переносы

%%%Цвет гиперссылок
\usepackage{color}
\definecolor{darkgreen}{rgb}{0,.5,0}
\usepackage[colorlinks,filecolor=blue,citecolor=darkgreen]{hyperref}

%%% Дополнительная работа с математикой
\usepackage{amsmath,amsfonts,amssymb,amsthm,mathtools} % AMS
\usepackage{icomma} % "Умная" запятая: $0,2$ --- число, $0, 2$ --- перечисление

%% Номера формул
%\mathtoolsset{showonlyrefs=true} % Показывать номера только у тех формул, на которые есть \eqref{} в тексте.
%\usepackage{leqno} % Нумерация формул слева

%% Свои команды
\DeclareMathOperator{\sgn}{\mathop{sgn}}

%% Перенос знаков в формулах (по Львовскому)
\newcommand*{\hm}[1]{#1\nobreak\discretionary{}
{\hbox{$\mathsurround=0pt #1$}}{}}

%%% Работа с картинками
\usepackage{graphicx}  % Для вставки рисунков
\graphicspath{{images/}{images2/}}  % папки с картинками
\setlength\fboxsep{3pt} % Отступ рамки \fbox{} от рисунка
\setlength\fboxrule{1pt} % Толщина линий рамки \fbox{}
\usepackage{wrapfig} % Обтекание рисунков текстом

%%% Работа с таблицами
\usepackage{array,tabularx,tabulary,booktabs} % Дополнительная работа с таблицами
\usepackage{longtable}  % Длинные таблицы
\usepackage{multirow} % Слияние строк в таблице

%%% Теоремы
\theoremstyle{plain} % Это стиль по умолчанию, его можно не переопределять.
\newtheorem{theorem}{Теорема}[section]
\newtheorem{proposition}[theorem]{Утверждение}
 
\theoremstyle{definition} % "Определение"
\newtheorem{corollary}{Следствие}[theorem]
\newtheorem{problem}{Задача}[section]
 
\theoremstyle{remark} % "Примечание"
\newtheorem*{nonum}{Решение}

%%% Программирование
\usepackage{etoolbox} % логические операторы

%%% Страница
%\usepackage{extsizes} % Возможность сделать 14-й шрифт
\usepackage{geometry} % Простой способ задавать поля
	\geometry{top=25mm}
	\geometry{bottom=35mm}
	\geometry{left=35mm}
	\geometry{right=20mm}
 %
\usepackage{fancyhdr} % Колонтитулы
 	\pagestyle{fancy}
 	\renewcommand{\headrulewidth}{0mm}  % Толщина линейки, отчеркивающей верхний колонтитул
 	%\lfoot{Нижний левый}
 	%\rfoot{Нижний правый}
 	%\rhead{Верхний правый}
 	%\chead{Верхний в центре}
 	%\lhead{Верхний левый}
 	% \cfoot{Нижний в центре} % По умолчанию здесь номер страницы

\usepackage{setspace} % Интерлиньяж
%\onehalfspacing % Интерлиньяж 1.5
%\doublespacing % Интерлиньяж 2
%\singlespacing % Интерлиньяж 1

\usepackage{lastpage} % Узнать, сколько всего страниц в документе.

\usepackage{soulutf8} % Модификаторы начертания

\usepackage{hyperref}
\usepackage[usenames,dvipsnames,svgnames,table,rgb]{xcolor}
\hypersetup{				% Гиперссылки
    unicode=true,           % русские буквы в раздела PDF
    pdftitle={Заголовок},   % Заголовок
    pdfauthor={Автор},      % Автор
    pdfsubject={Тема},      % Тема
    pdfcreator={Создатель}, % Создатель
    pdfproducer={Производитель}, % Производитель
    pdfkeywords={keyword1} {key2} {key3}, % Ключевые слова
    colorlinks=true,       	% false: ссылки в рамках; true: цветные ссылки
    linkcolor=red,          % внутренние ссылки
    citecolor=green,        % на библиографию
    filecolor=magenta,      % на файлы
    urlcolor=cyan           % на URL
}

%\renewcommand{\familydefault}{\sfdefault} % Начертание шрифта

\usepackage{multicol} % Несколько колонок

\author{Красильникова Виктория}
\title{Домашнее задание :
	Program HMM
	}
\date{9 марта 2019 г.}

\begin{document}
	
	\maketitle
	В ходе работы будут рассмотрены основные алгоритмы срытых Марковсих моделей. 
	\section{Определение скрытых марковских моделей}
	Скрытая Марковская модель или кратко HMM формально может быть описана следующим образом.
	\begin{itemize}
		\item 
		Множество дискретных состояний $S = (S_1,S_2,... S_M)$
		\item 
		Соостояние в момент времени $t: q_t$ 
		\item 
		Множество возможных сигналов $V = (V_1,V_2,... V_K)$
		\item
		Матрица переходов A из состояния  в состояние на мн-ве S, $A_{ij} = P(q_{t+1} = S_j | q_t = S_i)$
		\item
		Матрица B , которая содержит вероятности выдать сигнал в определенном состоянии 
		$B_{ij} = b_i(j) = P ( v_j | q_t = S_i )$
		\item
		Наблюдаемая последовательность сигналов $O=(O_1,O_2,...,O_L)$
		\item
		Вектор начального состояния $\pi , \pi_i = P(q_1 = S_i)$ 
		\item 
		HMM Model  $\lambda = 	(A,B, \pi)$
		
	\end{itemize}
	
	
	\section{HMM program}
	\subsection{Алгоритм Витерби}
		Реализуем алгоритм Viterbi для нахождения наиболее вероятной последовательности состояний данной HMM по последовательности наблюдаемых сигналов
	\\
	Дана последовательность $O = (O_1 , O_2, ... O_T)$ и модель $\lambda = (A,B, \pi)$	
	\\
	Необходимо подобрать последовательность состояний системы $ Q = q_1,q_2,...,q_T$, которая лучше всего соответствует наблюдаемой последовательности, т.е. объясняет последовательность наблюдений O.\\
	\subsection{Back-forward propagation}
	 Необходимо реализовать алгоритм Back-forward  для нахождения вероятности последовательности наблюдаемых сигналов для данной HMM.
	\\
	Таким образом дана последовательность  $O=(O_1,O_2,...,O_T)$ и модель $\lambda = 	(A,B, \pi)$ \\
	Необходимо вычислить вероятность $P(O,\lambda)$ что данная последовательность построена именно для нашей модели.
	
	Алгоритм состоит из 2-х частей \\
	1). Вычисление Forward значений 
	$ \alpha_t(i) = P(O_1, O_2 ,... ,O_t, q_t = S_i | \lambda) $ 
	\begin{itemize}
		\item 
		$\alpha_1(i) = \pi_i b_i(O_1)$
		\item
		$\alpha_{t+1} (j) =(\sum_{i=1}^{N}\alpha_t(i)a_{ij}) b_j(O_{t+1})$
		\item
		$P(O|\lambda) = \sum_{i=1}^{N}\alpha_T(i)$
	\end{itemize} 
	2). Вычисление Backward значений 
	$\beta_t(i) = P(O_{t+1},..., O_T, q_t = S_i | \lambda) $
	\begin{itemize}
	\item
	$\beta_T(i) = 1 $
	\item
	$ \beta_t(i) = \sum_{j=1}^{N}a_{ij}b_j(O_{t+1})\beta_{t+1}(j) $ 
	\item
	$P(O|\lambda) = \sum_{i=1}^{N} \beta_1(i)\pi_i b_i(O_1)$
	\end{itemize} 
	3). Вычисление постериорных вероятностей
	\begin{itemize}
		\item
		$ \gamma_t(i) = \frac{\alpha_t(i)\beta_t(i)}{P(O|\lambda)} $ 
		\item 
		${P(O|\lambda)} =  \sum_{i=1}^{N} \alpha_t(i) \beta_t(i) $
	\end{itemize} 
	\section{Выводы}
	В ходе выполнения работы были реализованы основные алгоритмы модели скрытых Марковских цепей.
	Также были вычислены наиболее вероятная последовательность состояний и параметры HMM на основе наблюдаемых сигналов в  предположении, что модель описывается эргодической Марковской цепью. \\
    Код исследования представлен в файле $hmm.ipynb$ \\
    Полученные значения параметров записаны в следующих файлах:
    \begin{itemize}
    	\item
    	Начальное распределение вероятностей  pi.csv
    	\item
    	Матрица переходов  A.csv
    	\item
    	Матрица эмиссий  B.csv
    \end{itemize}
    
		   
	
\end{document}